\subsection{Question 5}

\subsubsection{Lemmas}
\begin{verbatim}
lemma L(x1 : T, x2 : T, . . . , xN : T)
        requires P
        ensures R
{ }
\end{verbatim}

Lemmas can be called in a method to \textbf{prove} the lemmas property from that point onwards.

\subsubsection{Calc}
To prove a lemma by hand, you can add a \textbf{calc} section into the lemmas body, where \textit{$\gamma$} is the default transitive operator between lines.
\begin{Verbatim}[commandchars=\\\{\}, codes={\catcode`$=3\catcode`^=7}]
\textbf{calc} $\gamma$ \{
        5 * (x + 3);
        == 5 * x + 5 * 3;
        == 5x + 15;
\}
\end{Verbatim}

You can use use any transitive operator between lines (e.g. $==>$). If no default operator is specified, the default is $==$.

The \textbf{calc} statements can also be added inline within a method instead of creating and calling a lemma.

\subsubsection{Induction}
Lemmas can also be used to prove using induction by recursively calling the lemma in the body. E.g. \\
\begin{Verbatim}
lemma SumLemma(a: array<int>, i: int, j: int)
        requires P
        ensures R
{
        if i == j {
                // base case: Dafny can prove
        }
        else { 
                // inductive case
                SumLemma(a, i+1, j); 
        }
} 
\end{Verbatim}
