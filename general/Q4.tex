\subsection{Question 4}
\subsection{Classes}
Ghost variables can be used for specification and reasoning only.
\begin{Verbatim}[commandchars=\\\{\}]
    ghost var \textit{d}: \textit{T}
\end{Verbatim}

\subsubsection{Simple Classes}
A simple class consits of only simple object, (i.e. objects that are not stored on the heap).

The specification for a simple class consists of:
\begin{itemize}
    \item ghost variables for abstract state
    \item have class invariant, \textbf{ghost predicate Valid()}
    \item Valid() and functions have \textbf{reads this}
    \item constructor has \textbf{ensures Valid()}
    \item methods have \textbf{requires Valid()}, \textbf{modifies this}, \textbf{ensures Valid()}
\end{itemize}

Concrete states that consist of only simple objects are created and are related to the abstract state in \textbf{valid()}.

The constructor, methods, and functions must satisfy the class specification and will require both concrete and abstract state to be updated.

\subsubsection{Complex Classes}
Complex classes consist of any combination of simple and complex objects, (i.e. objects that are stored on the heap).

Complex classes require a representation set,
\begin{Verbatim}[commandchars=\\\{\}]
    \textbf{ghost var} Repr: set<object>
\end{Verbatim}

\subsubsubsection{Invariant}
The invariant valid will consist of the following, where a, a0, a1 are non-composite objects or arrays and b, b0, b1 are composite objects.
\begin{Verbatim}[commandchars=\\\{\}]
    \textbf{ghost predicate} Valid()
        \textbf{reads} this, Repr
        \textbf{ensures} Valid() ==> this \textbf{in} Repr
    \{ 
        this \textbf{in} Repr && ...
    \}
\end{Verbatim}

For a non-composite object or array \textbf{a}, include;
\begin{Verbatim}[commandchars=\\\{\}]
    a \textbf{in} Repr && a.Valid()
\end{Verbatim}

For a non-composite objects or arrays \textbf{a0, a1}, include;
\begin{Verbatim}[commandchars=\\\{\}]
    a0 != a1
\end{Verbatim}

For a composite object \textbf{b}, include;
\begin{Verbatim}[commandchars=\\\{\}]
    b \textbf{in} Repr && b.Repr <= Repr && 
    this !\textbf{in} b.Repr && b.Valid()
\end{Verbatim}

For a composite objects \textbf{b0, b1} and non-composite objects and arrays \textbf{a0, a1}, include;
\begin{Verbatim}[commandchars=\\\{\}]
    \{a0, a1\} !! b0.Repr !! b1.Repr  
\end{Verbatim}

\subsubsubsection{Constructor}
For a non-composite array or object \textbf{a} and a composite object \textbf{b}.
\begin{Verbatim}[commandchars=\\\{\}]
    \textbf{constructor}()
        \textbf{ensures} Valid() && \textbf{fresh}(Repr)
    \{
        ... \textit{(initialise concrete and abstract state)}
        \textbf{new};
        Repr := \{this, a, b\} + b.Repr;
    \}
\end{Verbatim}

\subsubsubsection{Functions}
\begin{Verbatim}[commandchars=\\\{\}]
    \textbf{function} F(x:X): Y()
        \textbf{requires} Valid()
        \textbf{reads} Repr
        \textbf{ensures} F(x) == \dots
\end{Verbatim}

\subsubsubsection{Methods (Mutating)}
\begin{Verbatim}[commandchars=\\\{\}]
    \textbf{method} M(x:X) returns Y()
        \textbf{requires} Valid()
        \textbf{modifies} Repr
        \textbf{ensures} Valid() && \textbf{valid}(Repr - \textbf{old}(Repr))
\end{Verbatim}