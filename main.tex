\documentclass[a4paper]{article}
\usepackage{multicol}
\usepackage{calc}
\usepackage{ifthen}
\usepackage[landscape]{geometry}
\usepackage{hyperref}
\usepackage{amsmath}
\usepackage{multicol}
\usepackage{fancyvrb}
\usepackage{enumitem}

% To make this come out properly in landscape mode, do one of the following
% 1.
%  pdflatex latexsheet.tex
%
% 2.
%  latex latexsheet.tex
%  dvips -P pdf  -t landscape latexsheet.dvi
%  ps2pdf latexsheet.ps


% If you're reading this, be prepared for confusion.  Making this was
% a learning experience for me, and it shows.  Much of the placement
% was hacked in; if you make it better, let me know...


% 2008-04
% Changed page margin code to use the geometry package. Also added code for
% conditional page margins, depending on paper size. Thanks to Uwe Ziegenhagen
% for the suggestions.

% 2006-08
% Made changes based on suggestions from Gene Cooperman. <gene at ccs.neu.edu>


% To Do:
% \listoffigures \listoftables
% \setcounter{secnumdepth}{0}


% This sets page margins to .5 inch if using letter paper, and to 1cm
% if using A4 paper. (This probably isn't strictly necessary.)
% % If using another size paper, use default 1cm margins.
\ifthenelse{\lengthtest { \paperwidth = 11in}}
	{ \geometry{top=.20in,left=.20in,right=.20in,bottom=.20in} }
	{\ifthenelse{ \lengthtest{ \paperwidth = 297mm}}
		{\geometry{top=1cm,left=1cm,right=1cm,bottom=1cm} }
		{\geometry{top=1cm,left=1cm,right=1cm,bottom=1cm} }
	}

% Turn off header and footer
\pagestyle{empty}
 

% Redefine section commands to use less space
\makeatletter
\renewcommand{\section}{\@startsection{section}{1}{0mm}%
                                {-1ex plus -.5ex minus -.2ex}%
                                {0.5ex plus .2ex}%
                                {\normalfont\normalsize\bfseries}}
\renewcommand{\subsection}{\@startsection{subsection}{2}{0mm}%
                                {-1explus -.5ex minus -.2ex}%
                                {0.5ex plus .2ex}%
                                {\normalfont\small\bfseries}}
\renewcommand{\subsubsection}{\@startsection{subsubsection}{3}{0mm}%
                                {-1ex plus -.5ex minus -.2ex}%
                                {1ex plus .2ex}%
                                {\normalfont\footnotesize\bfseries}}                   
\newcommand{\subsubsubsection}{\@startsection{subsubsection}{4}{0mm}%
                                {-1ex plus -.5ex minus -.2ex}%
                                {1ex plus .2ex}%
                                {\normalfont\scriptsize\bfseries}}
\makeatother
\newcommand{\imp}{\Rightarrow}

% Define BibTeX command
\def\BibTeX{{\rm B\kern-.05em{\sc i\kern-.025em b}\kern-.08em
    T\kern-.1667em\lower.7ex\hbox{E}\kern-.125emX}}

% Don't print section numbers
\setcounter{secnumdepth}{0}


\setlength{\parindent}{0pt}
\setlength{\parskip}{0pt plus 0.2ex}

\newcommand\tab[1][0.1cm]{\hspace*{#1}}

% -----------------------------------------------------------------------

\begin{document}

\raggedright
\fontsize{5pt}{6pt}\selectfont
\begin{multicols}{6}


% multicol parameters
% These lengths are set only within the two main columns
%\setlength{\columnseprule}{0.25pt}
\setlength{\premulticols}{1pt}
\setlength{\postmulticols}{1pt}
\setlength{\multicolsep}{1pt}
\setlength{\columnsep}{2pt}

listparameters=\setlength{\topsep}{1pt}\setlength{\partopsep}{2pt}

\begin{center}
     \Large{\textbf{CSSE3100 Crib Sheet}} \\
\end{center}

\subsection{Collection Types}

\setlength{\tabcolsep}{0pt} 

\subsubsection{Arrays}
Arrays are a mutable collection of elements stored on the heap.

For an array, \verb!a!, to be modified by a method, it must include
\verb!modifies a! in the specification.

\begin{tabular}{r@{\hspace{0.1cm}}l@{}}
        Type & \verb!array<T>!\\
        Creation & \verb!var a := new T[length];!\\
        Accessing & \verb!var value := a[i];!\\
        Assigning & \verb!a[i] := value;!\\
        Alias & \verb!var b := a;!\\
        Length & \verb!var l := a.Length;!\\
        Slicing & \verb!var s: seq<T> := a[start...end];!
\end{tabular}
\subsubsubsection{Multi-dimensional Arrays}
\begin{tabular}{r@{\hspace{0.1cm}}l@{}}
        Type & \verb!array<array<...<array<T>>>>!\\
        Creation & \verb!var a := new T[l1, l2, ..., lN];!\\
        Accessing & \verb!var value := a[i1, i2, ..., iN];!\\
        Asigning & \verb!a[i1, i2, ..., iN] := value;!\\
        Alias & \verb!var b := a;!\\
        Length & \verb!(l1, l2, ..., lN) := (a.Length1,! \\
                & \verb!        a.Length2, ... a.LengthN);!
\end{tabular}

\subsubsection{Sequences}
Sequences are used to represent an ordered list. They are immutable.
\begin{tabular}{r@{\hspace{0.1cm}}l@{}}
        Type & \verb!seq<T>!\\
        Creation & \verb!var s := [x1, x2, ..., xN];!\\
        Accessing & \verb!var value := s[i];!\\
        Length & \verb!var l := |s|;!\\
        Slicing & \verb!var t := a[start...end];!\\
        Appending & \verb!var u := s + t;!\\
        Contains & \verb!value in s;!\\
        Excludes & \verb|value !in s;|
\end{tabular}

\subsubsection{Sets}
Sets are used to represent an orderless collection of elements, without repetition. Sets are immutable.
\begin{tabular}{r@{\hspace{0.1cm}}l@{}}
        Type & \verb!set<T>!\\
        Creation & \verb!var s := {x1, x2, ..., xN};!\\
        Equality & \verb!{x1, x2} == {x2, x1} ==!\\
                & \verb!        {x1, x1, x2, x2};!\\
        Subset & \verb!s <= t!\\
        Proper Subset & \verb!s < t!\\
        Union & \verb!var u := s + t;!\\
        Intersection & \verb!var u := s * t;!\\
        Difference & \verb!var u := s - t;!\\
        Contains & \verb!elem in s;!\\
        Excludes & \verb|elem !in s;|
\end{tabular}

\subsubsection{Multisets}
Multisets are used to represent an orderless collection of elements, with repetition. Multisets are immutable.
\begin{tabular}{r@{\hspace{0.1cm}}l@{}}
        Type & \verb!multiset<T>!\\
        Creation & \verb!var s := multiset{x1, ..., xN};!\\
        From seq & \verb!var s := multiset([x1, ..., xN]);!\\
        From set & \verb!var s := multiset({x1, ..., xN});!\\
        Equality & \verb!multiset{x1, x2} == !\\
                & \verb|        multiset{x2, x1} !=|\\
                & \verb!        multiset{x1, x1, x2, x2};!\\
        Union & \verb!var u := s + t;!\\
        Difference & \verb!var u := s - t;!\\
        Contains & \verb!elem in s;!\\
        Excludes & \verb|elem !in s;|\\
        Disjoint & \verb|elem !! s;|
\end{tabular}

\subsection{Specification Keywords}
\subsubsection{Requires}
A requires clause stipulates a condition \verb!P! which must be true upon entry to the method.
\subsubsection{Ensures}
A requires clause stipulates a condition \verb!R! which must be true when exiting the method.
\subsubsection{Modifies}
A modifies clause is required if a method changes a value on the heap (i.e. a value in an array is changed).
\subsubsection{Reads}
A reads clause is required if a method reads a value on the heap (i.e. a value in an array is read).
\subsubsection{Invariant}
An invariant clause stipulates a condition \verb!I! which must be true at the beginning and end of a loop.
\subsubsection{Decreases}
A decreases clause indicates a value \verb!D! which decreases after every iteration of a loop.
\subsubsection{Forall}
A forall clause is used to stipulate that a condition \verb!Q! must hold forall values of a given variable.
For example, \verb!forall i :: P[i] ==> Q[i]! requires \verb!Q[i]! to hold for all values of \verb!i! where \verb!P[i]! holds.
\subsubsection{Exists}
An exists clause is used to stipulate that a condition \verb!Q! must hold for at least one value of a given variable.
For example, \verb!exists i :: P[i] ==> Q[i]! requires \verb!Q[i]! to hold for at least one value of \verb!i! where \verb!P[i]! holds.
\subsubsection{Fresh}
Fresh is used to indicate that a value stored on the heap must be brand new with no modifications.
For example, \verb!fresh(x)! requires x to be a brand new value on the heap.
\subsubsection{Old}
Old is used to reference the value on the heap before the method began. 
For example, \verb!old(a[i])! refers to the element at index \verb!i! at the beginning of the method.


\subsection{Question 1}

\subsubsection{Predicate Logic}
\setlength{\tabcolsep}{0pt} 
\begin{tabular}{@{}ll@{}}
$ A \land (A \lor B) \equiv A \equiv A \lor (A \land B)$ & \verb! (A.6)! \\
$ A \land (B \lor C) \equiv (A \land B) \lor (A \land C)$ & \verb! (A.7)! \\
$ A \lor (B \land C) \equiv (A \lor B) \land (A \lor C)$ & \verb! (A.8)! \\
$ \neg (A \land B) \equiv \neg A \lor \neg B$ & \verb! (A.18)! \\
$ \neg (A \lor B) \equiv \neg A \land \neg B$ & \verb! (A.19)! \\
$ A \lor (\neg A \land B) \equiv A \lor B$ & \verb! (A.20)! \\
$ A \land (\neg A \lor B) \equiv A \land B$ & \verb! (A.21)! \\
$ A \imp B \equiv \neg A \lor B$ & \verb! (A.22)! \\
$ A \imp B \equiv \neg(A \land \neg B)$ & \verb! (A.24)! \\
$ \neg(A \imp B) \equiv A \land \neg B$ & \verb! (A.25)! \\
$ A \imp B \equiv \neg B \imp \neg A$ & \verb! (A.26)! \\
$ C \imp (A \land B) \equiv (C \imp A) \land (C \imp B)$ & \verb! (A.33)! \\
$ (A \lor B) \imp C \equiv (A \imp C) \land (B \imp C)$ & \verb! (A.34)! \\
$ C \imp (A \lor B) \equiv (C \imp A) \lor (C \imp B)$ & \verb! (A.35)! \\
$ (A \land B) \imp C \equiv (A \imp C) \lor (B \imp C)$ & \verb! (A.36)! \\
$ A \imp (B \imp C) \equiv (A \land B) \imp C \equiv$ & \verb! (A.37)! \\
$ B \imp (A \imp C)$ & \\
$ (A \imp B) \land (\neg A \imp C) \equiv$ & \verb! (A.38)! \\
$ (A \land B) \lor (\neg A \land C)$ & \\
$ (\forall x \text{ s.t. } x = E \imp A) \equiv A[x \backslash E] \equiv $ & \verb! (A.56)! \\
$ (\exists x \text{ s.t. } x = E \land A)$ & \\
$ \forall x :: A \land B = (\forall x :: A) \land (\forall x :: B)$ & \verb! (A.65)! \\
$ \forall x :: A = A$ & \verb! (A.74)! \\
$ \text{ provided } x \text{ not free in } A$ & \\
\end{tabular}

\subsubsection{Rules to know}

\subsubsubsection{Basic Function} 
\begin{verbatim}
method MyMethod(x: int) returns (y: int)
    requires x == 10
    ensures y >= 25
{
    {x == 10}
    {x + 3 + 12 == 25}
    var a := x + 3;
    {a + 12 == 25}
    var b := 12;
    {a + b == 25}
    y := a + b;
    {y >= 25}
}

\end{verbatim}


\subsubsubsection{Loops} 
\begin{verbatim}
{J}
while B
        invariant J
{
        {B && J}
        ... 
        {J}
}
{J && !B}
\end{verbatim}

\begin{verbatim}
{y >= 4 && z >= x}
while z < 0
        invariant y >= 4 && z >= x
{
        {z < 0 && y >= 4 && z >= x}
        {y >= 4 && z + y >= x}
        z := z + y;
        {y >= 4 && z >= x}
}
{z >= 0 && y >= 4 && z >= x}
\end{verbatim}

\subsubsubsection{Methods} 
\begin{verbatim}
wp(t := M(E), Q )
  = P[x\E]
    &&  forall y' ::
      R[x,y\E, y'] 
        ==> Q[t\y']
\end{verbatim}
\begin{verbatim}
Given:
method Triple(x: int) returns (y: int)
requires x >= 0
ensures y == 3*x {}


{ u == 15}
{ u + 3 >= 0 &&
        3*(u + 3) == 54 } (A.56)
{ u + 3 >= 0 &&
        forall y' :: y' == 3*(u + 3)
                ==> y' == 54 }
t := Triple(u + 3);
{ t == 54 }

\end{verbatim}
\begin{verbatim}
function SeqSum(s: seq<int>, lo: int, hi: int): int
requires 0 <= lo <= hi <= |s|
decreases hi - lo
{
        if lo == hi then 0 else s[lo] +
            SeqSum(s, lo + 1, hi)
}
\end{verbatim}

\subsection{Question 2}
\subsubsection{Loop Design Techniques}
\subsubsubsection{Look in the postcondition.}
For a postcondition A \&\& B,
choose the invariant to be A and the guard to be !B.
\begin{verbatim}
method SquareRoot(N: nat) returns (r: nat)
ensures r*r <= N && N < (r + 1)*(r + 1)
    { { 0 <= N }
    { 0*0 <= N}
    r := 0;
    { r*r <= N }
    while (r + 1)*(r + 1) <= N
    invariant r*r <= N
    {
        { (r + 1)*(r + 1) <= N 
                && r*r <= N } (strengthen)
        { (r + 1)*(r + 1) <= N }
        r := r + 1;
        { r*r <= N }
    }
}
\end{verbatim}

\subsubsubsection{Programming by wishing}
If a problem can be made simpler by having a
precomputed quantity Q, then introduce a new
variable q with the intention of establishing and
maintaining the invariant q == Q

\begin{verbatim}
method SquareRoot(N: nat) returns (r: nat)
ensures r*r <= N < (r + 1)*(r + 1)
{
    r := 0;
    var s := 1;
    while s <= N
    invariant r*r <= N
    invariant s == (r + 1)*(r + 1)
    {
        s := s + 2*r + 3;
        r := r + 1;
    }
}
\end{verbatim}

\subsubsubsection{Replace a constant by a variable}
For a loop to establish a condition P(C), where C is an
expression that is held constant throughout the loop,
use a variable k that the loop changes until it equals C,
and make P(k) a loop invariant.

For example, Min method (Week 4) had postcondition
\begin{verbatim}
    ensures forall i :: 0 <= i < a.Length 
            ==> m <= a[i]
\end{verbatim}
and invariant
\begin{verbatim}
    invariant forall i :: 0 <= i < n 
            ==> m <= a[i]
\end{verbatim}

\subsubsubsection{What's yet to be done}.
If you're trying to solve a problem of the form
p == F(n), replacement of a constant by a variable
results in a what-has-been-done invariant
\begin{verbatim}
    invariant p == F(i)
\end{verbatim}
Alternatively, you may use a what's-yet-to-be-done
invariant

\begin{verbatim}
    invariant p @ F(n - i) == F(n)
\end{verbatim}
where @ is some kind of combination operation.

\subsubsubsection{Use the postcondition}
To establish a postcondition Q, make Q a loop
invariant.

For the Min example, to ensure the postcondiVon
\begin{verbatim}
ensures exists i :: 0 <= i < a.Length && 
                                m == a[i]
\end{verbatim}
we used the invariant
\begin{verbatim}
invariant exists i :: 0 <= i < a.Length && 
                                m == a[i]
\end{verbatim}

\subsection{Question 3}
\subsubsection{Termination Metrics}
Any set of values which have a \textit{well-founded} order can be used as a termination metric.

An order $\succ$ is well-founded when
\begin{itemize}
    \item $\succ$ is irreflexive: a $\succ$ a never holds
    \item $\succ$ is transitive:\\
        \tab a $\succ$ b \&\& b $\succ$ c $\implies$ a $\succ$ c
    \item there is no infinite descending chain\\
        \tab $\text{a}_1 \succ \text{a}_2  \succ \text{a}_3  \succ \dots$
\end{itemize}

We write X decreases to x as $X \succ x$.

For integers, $X \succ x$ when X > x \&\& X >= 0. \\
For booleans, $X \succ x$ when X \&\& !x.

A termination metric for a recursive function is a metric that can be proven to decrease every iteration.

E.g. for the function;
\begin{verbatim}
    function F(x: int): int 
    { 
        if x < 10 then x else F(x – 1)
    }
\end{verbatim}
the termination metric would be x since $x \succ x - 1$.

\subsubsubsection{Lexicographic tuples}
A lexicographic order is a component-wise comparison where earlier components are more significant.

$\{\text{a}_0, \text{a}_1, \text{a}_2, \dots, \text{a}_n\} \succ \{\text{b}_0, \text{b}_2, \text{b}_3, \dots, \text{b}_n\}$ if and only if\\
$\text{a}_0 \succ \text{b}_0\ ||\ (\text{a}_0 == \text{b}_0$ \&\& $\text{a}_1 \succ \text{b}_1)\ ||$\\
$\tab (\text{a}_0 == \text{b}_0$ \&\& $\text{a}_1 == \text{b}_1$ \&\&\\
$\tab \tab \text{a}_2 \succ \text{b}_2)\ || \dots\ ||$\\
$\tab (\text{a}_0 == \text{b}_0$ \&\& $\text{a}_1 == \text{b}_1$ \&\& $\dots$ \&\&\\
$\tab \tab \text{a}_{n-1} == \text{b}_{n-1}$ \&\& $\text{a}_n \succ \text{b}_n)$

A lexicographic ordering allows tuples to be used as termination metrics. 

\subsubsubsection{Mutually Recursive Functions}
Tuples can be used to provide termination metrics for mutually recursive functions since you can provide multiple values that the functions may reduce on.

E.g. for the following methods;
\begin{verbatim}
    method F(i: nat) returns (r: nat) { 
        if i <= 2 { r := 1; } 
        else {
            var h := H(i - 2);
            r := 1 + h; 
        } 
    } 

    method H(i: nat) returns (r: nat) { 
        if i == 0 { r := 0; } 
        else { 
            var f := F(i); 
            var h := H(i - 1); 
            r := f + h; 
        } 
    }
\end{verbatim}
the termination matrix would be \{i, 1\} for H and \{i, 0\} for F since the call F(i) in H will reduce on 1 $\succ$ 0. 

\subsection{Question 4}
\subsection{Classes}
Ghost variables can be used for specification and reasoning only.
\begin{Verbatim}[commandchars=\\\{\}]
    ghost var \textit{d}: \textit{T}
\end{Verbatim}

\subsubsection{Simple Classes}
A simple class consits of only simple object, (i.e. objects that are not stored on the heap).

The specification for a simple class consists of:
\begin{itemize}
    \item ghost variables for abstract state
    \item have class invariant, \textbf{ghost predicate Valid()}
    \item Valid() and functions have \textbf{reads this}
    \item constructor has \textbf{ensures Valid()}
    \item methods have \textbf{requires Valid()}, \textbf{modifies this}, \textbf{ensures Valid()}
\end{itemize}

Concrete states that consist of only simple objects are created and are related to the abstract state in \textbf{valid()}.

The constructor, methods, and functions must satisfy the class specification and will require both concrete and abstract state to be updated.

\subsubsection{Complex Classes}
Complex classes consist of any combination of simple and complex objects, (i.e. objects that are stored on the heap).

Complex classes require a representation set,
\begin{Verbatim}[commandchars=\\\{\}]
    \textbf{ghost var} Repr: set<object>
\end{Verbatim}

\subsubsubsection{Invariant}
The invariant valid will consist of the following, where a, a0, a1 are non-composite objects or arrays and b, b0, b1 are composite objects.
\begin{Verbatim}[commandchars=\\\{\}]
    \textbf{ghost predicate} Valid()
        \textbf{reads} this, Repr
        \textbf{ensures} Valid() ==> this \textbf{in} Repr
    \{ 
        this \textbf{in} Repr && ...
    \}
\end{Verbatim}

For a non-composite object or array \textbf{a}, include;
\begin{Verbatim}[commandchars=\\\{\}]
    a \textbf{in} Repr && a.Valid()
\end{Verbatim}

For a non-composite objects or arrays \textbf{a0, a1}, include;
\begin{Verbatim}[commandchars=\\\{\}]
    a0 != a1
\end{Verbatim}

For a composite object \textbf{b}, include;
\begin{Verbatim}[commandchars=\\\{\}]
    b \textbf{in} Repr && b.Repr <= Repr && 
    this !\textbf{in} b.Repr && b.Valid()
\end{Verbatim}

For a composite objects \textbf{b0, b1} and non-composite objects and arrays \textbf{a0, a1}, include;
\begin{Verbatim}[commandchars=\\\{\}]
    \{a0, a1\} !! b0.Repr !! b1.Repr  
\end{Verbatim}

\subsubsubsection{Constructor}
For a non-composite array or object \textbf{a} and a composite object \textbf{b}.
\begin{Verbatim}[commandchars=\\\{\}]
    \textbf{constructor}()
        \textbf{ensures} Valid() && \textbf{fresh}(Repr)
    \{
        ... \textit{(initialise concrete and abstract state)}
        \textbf{new};
        Repr := \{this, a, b\} + b.Repr;
    \}
\end{Verbatim}

\subsubsubsection{Functions}
\begin{Verbatim}[commandchars=\\\{\}]
    \textbf{function} F(x:X): Y()
        \textbf{requires} Valid()
        \textbf{reads} Repr
        \textbf{ensures} F(x) == \dots
\end{Verbatim}

\subsubsubsection{Methods (Mutating)}
\begin{Verbatim}[commandchars=\\\{\}]
    \textbf{method} M(x:X) returns Y()
        \textbf{requires} Valid()
        \textbf{modifies} Repr
        \textbf{ensures} Valid() && \textbf{valid}(Repr - \textbf{old}(Repr))
\end{Verbatim}

\subsection{Question 5}

\subsubsection{Lemmas}
\begin{verbatim}
lemma L(x1 : T, x2 : T, . . . , xN : T)
        requires P
        ensures R
{ }
\end{verbatim}

Lemmas can be called in a method to \textbf{prove} the lemmas property from that point onwards.

\subsubsection{Calc}
To prove a lemma by hand, you can add a \textbf{calc} section into the lemmas body, where \textit{$\gamma$} is the default transitive operator between lines.
\begin{Verbatim}[commandchars=\\\{\}, codes={\catcode`$=3\catcode`^=7}]
\textbf{calc} $\gamma$ \{
        5 * (x + 3);
        == 5 * x + 5 * 3;
        == 5x + 15;
\}
\end{Verbatim}

You can use use any transitive operator between lines (e.g. $==>$). If no default operator is specified, the default is $==$.

The \textbf{calc} statements can also be added inline within a method instead of creating and calling a lemma.

\subsubsection{Induction}
Lemmas can also be used to prove using induction by recursively calling the lemma in the body. E.g. \\
\begin{Verbatim}
lemma SumLemma(a: array<int>, i: int, j: int)
        requires P
        ensures R
{
        if i == j {
                // base case: Dafny can prove
        }
        else { 
                // inductive case
                SumLemma(a, i+1, j); 
        }
} 
\end{Verbatim}

\subsection{Functional Programming}
Key features:
\begin{itemize}
        \item Program structures as mathematical functions
        \item Data is immutable (i.e. no heap, no side effects)
\end{itemize}

\subsubsection{Match}

\textbf{Match} is dafny's version of a switch statement, but it must cover all cases.
\begin{verbatim}
match x
case c1
case c2
. . .
case cn
\end{verbatim}

\subsubsection{Descriminators}
Discriminators can be used to check if a variable is a given type. E.g. \verb!xs.Nil?! checks if xs is type Nil.

\subsubsection{Destructors}
Destructors are used to access data in a composite datatype. E.g. for a variable xs of the datatype\\
\verb!datatype List<T> =!\\
\verb!          Nil | Cons(head: T, tail: List<T>)!,\\
head can be accessed using \verb!xs.head!. Similarly tail can be accessed using \verb!xs.tail!.

\subsubsection{Instrinsic vs Extrinsic Property}
\begin{itemize}
        \item An intrinsic property is a property defined within a specification.
        \item An extrinsic property is a property defined externally using a lemma.
        \item Methods in Dafny are opaque, so all properties in the specification are intrinsic.
        \item Functions are transparent, so properties can be intrinsic or extrinsic.
        \item Intrinsic properties are available every time we apply a function, whereas extrinsic properties are only available if we call the lemma.
        \item Having all properties exposed instrinsicly can lead to long verification times, so only define properties intrinsicly if they will be required for all applications of the function.
\end{itemize}

\section{2023 Final Exam}

\subsubsection{Question 1}
Provide weakest precondition proofs to determine whether or not the following methods
satisfy their specifications.
\subsubsubsection{(a)}
\begin{verbatim}
method M(x: int) returns (r: int)
  requires x >= -2
  ensures r >= 1
{
    { x == -2 || x >= 0 }
    { x + 1 == -1 || x + 1 >= 1 }
    r := x + 1;
    { r == -1 || r >= 1}
    { (r < 0 && r >= -1) || (r >= 0 && r >= 1) }
    { (r < 0 ==> r >= -1) && (r >= 0 ==> r >= 1) }
    if r < 0 {
        { r >= -1 }
        { r + 2 >= 1 }
        r := r + 2;
        { r >= 1}
    }
    { r >= 1}
}

Not correct since !(x >= -2 ==> x == -2 || x >= 0)
        since x >= -2 allows x to be -1.
\end{verbatim}
\subsubsubsection{(b)}
\begin{verbatim}
method B(x: int, y: int) returns (r: int)
requires x >= 0 && y >= 0
ensures r == x * y


method A(x: int, y: int) returns (r: int)
  requires y >= 4
  ensures r >= x + y
{ 
  {y >= 4}
  {y >= 4 && x == x}
  {y >= 4 && x >= x}
  var z := x;
  {y >= 4 && z >= x}
  while z < 0
    invariant y >= 4 && z >= x
  {
    {y >= 4 && z >= x && z < 0}
    {y >= 4 && z + y >= x && z < 0} (Strengthening)
    {y >= 4 && z + y >= x}
    {y >= 4 && z + y >= x}
    z := z + y;
    {y >= 4 && z >= x}
  }
  {z >= 0 && y >= 4 && z >= x} (Strengthening)
  {z >= 0 && y - 1 >= 0 && z * y - 1 >= x} (A.56)
  {z >= 0 && y - 1 >= 0 && forall y' :: y'
                        == z * y - 1 ==> y' >= x}
  r := B(z, y - 1);
  { r >= x}
  { r + y >= x + y}
  r := r + y;
  {r >= x + y}
}

Correct since y >= 4 ==> y >= 
\end{verbatim}

\subsubsection{Question 2}
\subsubsubsection{(a)}
Write a specification for a Dafny method to reverse an array. For example, given the array
[1, 2, 3, 4, 5] the method will change it to [5, 4, 3, 2, 1]. Note that the method should modify an
existing array, not create a new one.

\begin{verbatim}
method Reverse(a: array)
  modifies a
  ensures forall i :: 0 <= i < a.Length ==>
                        a[i] == old(a[a.Length-1-i])
\end{verbatim}

\subsubsubsection{(b)}

Based on your specification, provide a loop specification (guard and invariant) for the
Reverse method, and code to initialise the loop variables.

\begin{verbatim}
var n := 0;
while n < a.Length/2
  invariant 0 <= n <= a.Length/2
  invariant forall i :: 0 <= i < n
                ==> a[i] == old(a[a.Length-1-i])
  invariant forall i :: a.Length-n <= i < a.Length
                ==> a[i] == old(a[a.Length-1-i])
  invariant forall i :: n <= i < a.Length-n 
                ==> a[i] == old(a[i])
\end{verbatim}
The second and third invariants are instances of the Replace a Constant by a Variable loop
design technique. In the second invariant, the constant a.Length is replaced by n. In the third
invariant, the constant 0 is replaced by a.Length-n.
The final invariant states that nothing between indices n and a.Length-n have been changed by
the loop. This is similar to the additional invariant we required for the IncrementArray example
in Week 5.

\subsubsubsection{(c)}
Provide a termination metric for the loop.
\begin{verbatim}
decreases a.Length/2 - n
\end{verbatim}

\subsubsection{Question 3}

Provide termination metrics for the following mutually recursive methods

\begin{verbatim}
method F(i: nat) returns (r: nat) {
  if i <= 2 {
    r := 1;
  } else {
    var h := H(i - 2);
    r := 1 + h;
  }
}
method H(i: nat) returns (r: nat) {
  if i == 0 {
    r := 0;
  } else {
    var f := F (i);
    var h : = H(i - 1);
    r := f + h;
  }
}
\end{verbatim}

Justify your choice of termination metrics using the fact that an integer value X decreases to x
when X > x \& X => 0


Call H from F
  $i, 1 \succ i -2, 1$ \newline
Call F from H
  $i, 1 \succ i, 0$ \newline
Call H from H
  $i, 1 \succ i - 1, 0$ \newline

F decreases $i, 0$ \newline
H decrease $i, 1$


\subsubsection{Question 4}
\subsubsection{(a)}
Provide variable declarations representing the abstract and concrete states of the class.
Assume that the class has a generic parameter Event corresponding to the event type


\begin{verbatim}
// abstract
ghost var schedule: seq<Event>
ghost var additions: seq<Event>
ghost const n: nat
ghost var Repr: set<object>
// concrete
var events: array<Event>
var m: int
var n: int
\end{verbatim}

\subsubsection{(b)}
Provide a class invariant, Valid, for the class.

\begin{verbatim}
ghost predicate Valid( )
    reads this, Repr
    ensures Valid( ) ==> this in Repr 
                && |schedule| + |additions| <= N &&
    forall i, j :: 0 < = i < j
                < |schedule+additions| ==>
    (schedule + additions)[i]
                != (schedules + additions)[j]
{
    this in Repr && a in Repr &&
    0 <= m <= n <= a.Length && a.Length == N &&
    a[..m] == schedule && a[m..n] == additions &&
    forall i, j :: 0 <= i < j < n ==> a[i] != a[j]
}
\end{verbatim}

\subsubsection{(c)}

\begin{verbatim}
constructor (N : int)
    ensures Valid( ) && fresh(Repr)
    ensures schedule == [ ] && additions == [ ]
            && this.N == N
method AddEvent(e: Event)
    requires Valid( ) && e !in schedule
            && e !in additions 
            && |schedule + additions| < N
    modifies Repr
    ensures Valid( ) && fresh(Repr - old(Repr))
    ensures additions == old(additions) + [e]
            && schedule == old(schedule)
method Commit( )
    requires Valid( )
    modifies Repr
    ensures Valid( ) && fresh(Repr - old(Repr))
    ensures additions == [] && schedule ==
            old(schedule + additions)
method Abort( )
    requires Valid( )
    modifies Repr
    ensures Valid( ) && fresh(Repr - old(Repr))
    ensures additions == []
            && schedule == old(schedule)
\end{verbatim}

\subsection{Question 5}

Recall the datatype definition of a list and function Length from the lectures.
\begin{verbatim}
datatype List<T> = Nil | Cons(head: T, tail: List<T>)
function Length<T>(xs: List<T>): nat {
        match xs
        case Nil => 0
        case Cons(_ , tail) => 1 + Length(tail)
}    
\end{verbatim}

\subsubsection{(a)}
Write a function Remove which takes a list and an index i of the list as arguments and
returns a new list with the element at index i removed. For example, given the list [0, 1, 2, 3] and
index 2, the function should return [0, 1, 3].

\begin{verbatim}
function Remove<T>(xs: List<T>, i: nat): List<T>
  requires i < Length(xs)
{
  match xs
  case Cons(x, tail) => if i == 0 then tail
        else Cons(x, Remove(tail, i-1))
}
\end{verbatim}

\subsubsection{(b)}
The length of the list returned by Remove is one less than the length of the list provided as
an argument. Show how this would be stated as an intrinsic property of Remove.

The following is added to the function above
\begin{verbatim}
  ensures Length(Remove(xs,i)) == Length(xs) - 1
\end{verbatim}

\subsubsection{(c)}
State the property of part (b) as an extrinsic property of Remove.
\begin{verbatim}
lemma LengthRemove<T>(xs: List<T>, i: nat)
  requires i < Length(xs)
  ensures Length(Remove(xs,i)) == Length(xs) - 1
\end{verbatim}

\subsection{Tut 10.3}
\begin{verbatim}
class Node<T> {
  ghost var s: seq<T>
  ghost var Repr: set<object>
  // concrete state
  var value: T
  var next: Node?<T>

  ghost predicate Valid()
    reads this, Repr
    ensures Valid() ==> this in Repr && |s| > 0
  {
    this in Repr &&
    (next == null ==> s == [value]) &&
    (next != null ==> next in Repr && next.Repr <= Repr && this !in next.Repr &&
    next.Valid() && s == [value] + next.s)
  }

  constructor (v: T)
    ensures Valid() && fresh(Repr)
    ensures s == [v]
  {
    value := v;
    next := null;
    s, Repr := [v], {this};
  }

  method SetNext(n: Node<T>)
    requires Valid() && n.Valid() && this !in n.Repr && n.Repr !! Repr
    modifies Repr
    ensures Valid() && fresh(Repr - old(Repr) - n.Repr)
    ensures s == old([s[0]]) + n.s
  {
    next := n;
    s, Repr := [value] + n.s, Repr + next.Repr;
  }

  method GetNext() returns (n: Node?<T>)
    requires Valid()
    ensures n == null ==> |s| == 1
    ensures n != null ==> n in Repr && n.Repr <= Repr && this !in n.Repr &&
    n.Valid() && s == s[0] + n.s
  {
    n := next;
  }

  method GetValue() returns (v: T)
    requires Valid()
    ensures v == s[0]
  {
    v := value;
  }
}


class Stack<T> {
  ghost var s: seq<T>
  ghost var Repr: set<object>
  // concrete state
  var top: Node?<T>
  ghost predicate Valid()
  reads this, Repr
  ensures Valid() ==> this in Repr
{
  this in Repr &&
  (top == null ==> s == []) &&
  (top != null ==> top in Repr && top.Repr <= Repr && this !in top.Repr &&
  top.Valid() && top.s == s)
}

constructor ()
  ensures Valid() && fresh(Repr)
  ensures s == []
{
  top := null;
  s, Repr := [], {this};
}

method Push(v: T)
  requires Valid()
  modifies Repr
  ensures Valid() && fresh(Repr - old(Repr))
  ensures s == [v] + old(s)
{
  var newNode := new Node(v);
  if top != null {
  newNode.SetNext(top);
  }
  top := newNode;
  s, Repr := [v] + s, {this} + newNode.Repr;
}

method Pop() returns (v: T)
  requires s != []
  requires Valid()
  modifies Repr
  ensures Valid() && fresh(Repr - old(Repr))
  ensures v == old(s[0]) && s == old(s[1..])
{
  v := top.GetValue();
  top := top.GetNext();
  s := s[1..]; // note that the removal of old(top) from Repr is not required
}
}
\end{verbatim}

\scriptsize

\end{multicols}
\end{document}
