\section{Question 2}
\subsection{Loop Design Techniques}
\subsubsection{Look in the postcondition.}
For a postcondition A \&\& B,
choose the invariant to be A and the guard to be !B.
\begin{verbatim}
method SquareRoot(N: nat) returns (r: nat)
ensures r*r <= N && N < (r + 1)*(r + 1)
    { { 0 <= N }
    { 0*0 <= N}
    r := 0;
    { r*r <= N }
    while (r + 1)*(r + 1) <= N
    invariant r*r <= N
    {
        { (r + 1)*(r + 1) <= N 
                && r*r <= N } (strengthen)
        { (r + 1)*(r + 1) <= N }
        r := r + 1;
        { r*r <= N }
    }
}
\end{verbatim}

\subsubsection{Programming by wishing}
If a problem can be made simpler by having a
precomputed quantity Q, then introduce a new
variable q with the intention of establishing and
maintaining the invariant q == Q

\begin{verbatim}
method SquareRoot(N: nat) returns (r: nat)
ensures r*r <= N < (r + 1)*(r + 1)
{
    r := 0;
    var s := 1;
    while s <= N
    invariant r*r <= N
    invariant s == (r + 1)*(r + 1)
    {
        s := s + 2*r + 3;
        r := r + 1;
    }
}
\end{verbatim}

\subsubsection{Replace a constant by a variable}
For a loop to establish a condition P(C), where C is an
expression that is held constant throughout the loop,
use a variable k that the loop changes until it equals C,
and make P(k) a loop invariant.

For example, Min method (Week 4) had postcondition
\begin{verbatim}
    ensures forall i :: 0 <= i < a.Length ==>
                                        m <= a[i]
\end{verbatim}
and invariant
\begin{verbatim}
    invariant forall i :: 0 <= i < n ==> m <= a[i]
\end{verbatim}

\subsubsection{What's yet to be done}.
If you're trying to solve a problem of the form
p == F(n), replacement of a constant by a variable
results in a what-has-been-done invariant
\begin{verbatim}
    invariant p == F(i)
\end{verbatim}
Alternatively, you may use a what's-yet-to-be-done
invariant

\begin{verbatim}
    invariant p @ F(n – i) == F(n)
\end{verbatim}
where @ is some kind of combination operation.

\subsubsection{Use the postcondition}
To establish a postcondition Q, make Q a loop
invariant.

For the Min example, to ensure the postcondiVon
\begin{verbatim}
ensures exists i :: 0 <= i < a.Length && m == a[i]
\end{verbatim}
we used the invariant
\begin{verbatim}
invariant exists i :: 0 <= i < a.Length && m == a[i]
\end{verbatim}