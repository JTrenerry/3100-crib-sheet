\section{Question 3}
\subsection{Termination Metrics}
Any set of values which have a \textit{well-founded} order can be used as a termination metric.

An order $\succ$ is well-founded when
\begin{itemize}
    \item $\succ$ is irreflexive: a $\succ$ a never holds
    \item $\succ$ is transitive:\\
        \tab a $\succ$ b \&\& b $\succ$ c $\implies$ a $\succ$ c
    \item there is no infinite descending chain\\
        \tab $\text{a}_1 \succ \text{a}_2  \succ \text{a}_3  \succ \dots$
\end{itemize}

We write X decreases to x as $X \succ x$.

For integers, $X \succ x$ when X > x \&\& X >= 0. \\
For booleans, $X \succ x$ when X \&\& !x.

A termination metric for a recursive function is a metric that can be proven to decrease every iteration.

E.g. for the function;
\begin{verbatim}
    function F(x: int): int 
    { 
        if x < 10 then x else F(x – 1)
    }
\end{verbatim}
the termination metric would be x since $x \succ x - 1$.

\subsubsection{Lexicographic tuples}
A lexicographic order is a component-wise comparison wehre earlier components are more significant.

$\{\text{a}_0, \text{a}_1, \text{a}_2, \dots, \text{a}_n\} \succ \{\text{b}_0, \text{b}_2, \text{b}_3, \dots, \text{b}_n\}$ if and only if\\
$\text{a}_0 \succ \text{b}_0\ ||\ (\text{a}_0 == \text{b}_0$ \&\& $\text{a}_1 \succ \text{b}_1)\ ||$\\
$\tab (\text{a}_0 == \text{b}_0$ \&\& $\text{a}_1 == \text{b}_1$ \&\&\\
$\tab \tab \text{a}_2 \succ \text{b}_2)\ || \dots\ ||$\\
$\tab (\text{a}_0 == \text{b}_0$ \&\& $\text{a}_1 == \text{b}_1$ \&\& $\dots$ \&\&\\
$\tab \tab \text{a}_{n-1} == \text{b}_{n-1}$ \&\& $\text{a}_n \succ \text{b}_n)$

A lexicographic ordering allows tuples to be used as termination metrics. 

\subsubsection{Mutually Recursive Functions}
Tuples can be used to provide termination metrics for mutually recursive functions since you can provide multiple values that the functions may reduce on.

E.g. for the following methods;
\begin{verbatim}
    method F(i: nat) returns (r: nat) { 
        if i <= 2 { r := 1; } 
        else {
            var h := H(i - 2);
            r := 1 + h; 
        } 
    } 

    method H(i: nat) returns (r: nat) { 
        if i == 0 { r := 0; } 
        else { 
            var f := F(i); 
            var h := H(i - 1); 
            r := f + h; 
        } 
    }
\end{verbatim}
the termination matrix would be \{i, 1\} for H and \{i, 0\} for F since the call F(i) in H will reduce on 1 $\succ$ 0. 